% 第5節 相似な図形の面積比、体積比
\section{相似な図形の面積比、体積比}

ここまでの内容を踏まえると、相似な図形の面積比と体積比について考えることができる。

%面積比
    \begin{theorembox}[thm:相似な図形の面積比]{\textbf{相似な図形の面積比}}
        相似比が $\fitblankbf[m:n]{5}$ である相似な図形の面積比は $\fitblankbf[m^2:n^2]{8}$ である。
    \end{theorembox}
    \vspace{1em}

    Theorem\ref{thm:相似な図形の面積比}は三角形や四角形に限らず、
    \uwave{三角形や四角形以外の多角形や円などの一般の平面図形についても成り立つ。}

    また、相似という概念は平面だけでなく、立体についても考えることができる。

    %相似
    \begin{definitionbox}[def:相似な立体]{\textbf{相似な立体}}
        1つの立体を一定の割合に \fitblankbf[拡大]{5} または\fitblankbf[縮小]{5} 得られる立体は、
        もとの図形と \fitblankbf[相似]{5} であるという。
    \end{definitionbox}
    \vspace{1em}

    一言で言うと、大きさは \fitblankbf[異なる]{5} が、\fitblankbf[形は同じ]{8} の立体のことである。
    したがって、相似な立体の性質は以下の2つである。

    \begin{theorembox}[thm:相似な立体の性質]{\textbf{相似な立体の性質}}
        相似な立体の性質は以下の2つである。相似な立体において

    \begin{enumerate}
            \item 対応する \fitblankbf[線分の長さの比]{10} はすべて等しい。
            \item 対応する \fitblankbf[角の大きさ]{8} はそれぞれ等しい。
        \end{enumerate}
    \end{theorembox}
    \vspace{1em}

%相似比
    \begin{definitionbox}[def:相似比]{\textbf{相似比}}
        相似な立体においても、対応する線分の長さの比を \fitblankbf[相似比]{5} という。
    \end{definitionbox}
    \vspace{1em}

    次に、相似な立体の表面積の比、体積の比について考えてみよう。

    \begin{definitionbox}[def:表面積比と体積比]{\textbf{表面積比と体積比}}
        表面積の比を単に \fitblankbf[表面積比]{5} 、体積の比を単に \fitblankbf[体積比]{5} という。
    \end{definitionbox}
    \vspace{1em}

    一般に、相似な立体について、次のことが成り立つ。

%表面積比と体積比
    \begin{theorembox}[thm:相似な立体の体積比]{\textbf{相似な立体の体積比}}
        相似比が $\fitblankbf[m:n]{5}$ である相似な立体について
        表面積比は $\fitblankbf[m^2:n^2]{8}$ 、
        体積比は $\fitblankbf[m^3:n^3]{8}$ である。
    \end{theorembox}
    \vspace{1em}


\newpage

