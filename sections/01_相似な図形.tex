% 第1節 相似な図形
\section{相似な図形}

%相似
    \begin{definitionbox}[def:相似]{\textbf{相似}}
        2つの図形の一方の図形を \fitblankbf[拡大]{5} または\fitblankbf[縮小]{5} した図形が他方と合同になるとき、
        この2つの図形は \fitblankbf[相似]{5} であるという。
    \end{definitionbox}
    \vspace{1em}

    一言で言うと、大きさは \fitblankbf[異なる]{5} が、\fitblankbf[形は同じ]{8} の図形のことである。
    したがって、

    \begin{theorembox}[thm:相似な図形の性質]{\textbf{相似な図形の性質}}
        相似な図形の性質は以下の2つである。

    \begin{enumerate}
            \item 相似な図形では 対応する \fitblankbf[線分の長さの比]{10} はそれぞれ等しい。
            \item 相似な図形では 対応する \fitblankbf[角の大きさ]{8} はすべて等しい。
        \end{enumerate}
    \end{theorembox}
    \vspace{1em}

    \begin{definitionbox}[def:相似な図形の表し方]{\textbf{相似な図形の表し方}}
        2つの図形が相似であることを記号 $\fitblankbf[\sim]{3}$ を用いて表す。

        例えば、$\triangle ABC$ と $\triangle DEF$ が相似であることを $\fitblankbf[\triangle ABC \sim \triangle DEF]{18}$ と表し、
        「三角形 ABC 相似 三角形 DEF」と読む。
        また、記号 $\sim$ を用いるときは、頂点の順番を \fitblankbf[対応する順]{5} で書く。
    \end{definitionbox}
    \vspace{1em}

%相似比
    \begin{definitionbox}[def:相似比]{\textbf{相似比}}
        相似な図形で、対応する線分の長さの比を \fitblankbf[相似比]{5} という。
    \end{definitionbox}
    \vspace{1em}

\newpage

