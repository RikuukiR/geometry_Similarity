% 第2節 三角形の相似条件
\section{三角形の相似条件}

特に三角形においては、相似になるための条件が知られている。これを \fitblankbf[相似条件]{8} といい、次の3つがある。
相似条件は証明中に使い、\uwave{一言一句違わずに} 使わなければならない。

%相似条件
    \begin{theorembox}[thm:相似条件]{\textbf{相似条件}}
        2つの三角形が相似であるための条件は以下の3つである。

        \begin{enumerate}[itemsep=1em]
            \item \fitblankfixed[3組の辺の比がすべて等しい]{30}
            \item \fitblankfixed[2組の辺の比とその間の角がそれぞれ等しい]{30}
            \item \fitblankfixed[2組の角がそれぞれ等しい]{30}
        \end{enumerate}
    \end{theorembox}
    \vspace{1em}

\textbf{注意:} 合同条件と相似条件の違いに注意すること。以下、合同条件を書きなさい。

\begin{enumerate}[itemsep=1em]
    \item \fitblankfixed[3組の辺がそれぞれ等しい]{30}
    \item \fitblankfixed[2組の辺の比とその間の角がそれぞれ等しい]{30}
    \item \fitblankfixed[2組の角がそれぞれ等しい]{30}
\end{enumerate}
\vspace{1em}
\newpage

証明の流れと型を身につけてもらいたい。学校の型と異なるかとは思うが、ここでは私がオススメする型を紹介する。
特にこだわりがない場合は、この型を是非使ってもらいたい。

ここで使用する流れや型は2年次で学習した「合同な図形」の証明においても使用した汎用的なものである。

%証明の流れ
    \begin{theorembox}[thm:証明の流れ]{\textbf{証明の流れ}}
        \label{thm:証明の流れ}
        証明の流れは以下のようになる。

        \begin{enumerate}
            \item \fitblankbf[証明したい図形]{10} を確認し、一方を\fitblankbf[赤い三角形]{8}で、もう一方を\fitblankbf[青い三角形]{8}で囲む。
            \item 赤い三角形と青い三角形を \fitblankbf[抜き出し]{8} $^{*}$、\fitblankbf[仮定]{8} や \fitblankbf[条件]{8} を描き込む。
            \item(2)で抜き出した図形をもとに \fitblankbf[相似条件]{8} を確定させる。$^{**}$
        \end{enumerate}
    \end{theorembox}
    \vspace{0.5em}

    Theorem \ref{thm:証明の流れ} で証明の下準備は完了!証明は書き始める前に終えてなければならない。
    したがって、Theorem \ref{thm:証明の流れ} を使って証明の型にはめていくことになる。
    \vspace{0.5em}

    \begin{theorembox}[thm:証明の型]{\textbf{証明の型}}
        \label{thm:証明の型}
        証明の型は以下のようになる。ここでは \textcolor{red}{\underline{$\triangle ABC$}} と
        \textcolor{blue}{\underline{$\triangle DEF$}} が相似であることを証明する。

        \begin{proof}
            \setlength{\baselineskip}{1.5\baselineskip}
            \mbox{}\\
            \textcolor{red}{\fitblankbf[\triangle ABC]{8}} と \textcolor{blue}{\fitblankbf[\triangle DEF]{8}} において

            \vspace{0.3em}
            \textcolor{black}{\uwave{\hspace{2em}根拠①\hspace{3em}}} より、

            \vspace{0.3em}
            \makebox[\linewidth][s]{\textcolor{red}{\fitblankbf[AB]{8}} = \textcolor{blue}{\fitblankbf[DE]{8}} \hfill ・・・①}

            \vspace{0.3em}
            \textcolor{black}{\uwave{\hspace{2em}根拠②\hspace{3em}}} ので、

            \vspace{0.3em}
            \makebox[\linewidth][s]{\textcolor{red}{\fitblankbf[\angle BAC]{8}} = \textcolor{blue}{\fitblankbf[\angle EDF]{8}} \hfill ・・・②}

            \vspace{0.3em}
            \textcolor{black}{\uwave{\hspace{2em}根拠③\hspace{3em}}} ので、

            \vspace{0.3em}
            \makebox[\linewidth][s]{\textcolor{red}{\fitblankbf[CA]{8}} = \textcolor{blue}{\fitblankbf[FD]{8}} \hfill ・・・③}

            \vspace{0.3em}
            ①、②、③より、\textcolor{black}{\fitblankbf[相似条件]{15}} ので、

            \vspace{0.3em}
            \textcolor{red}{\fitblankbf[\triangle ABC]{8}} $\sim$ \textcolor{blue}{\fitblankbf[\triangle DEF]{8}}
        \end{proof}
    \end{theorembox}
\vfill

\noindent
\rule{\textwidth}{0.4pt}

\noindent
$^{*}$ 図形を抜き出す際には \textbf{ 形は気にせず } 抜き出してもらって構わない。しかし、2つの三角形とも \textbf{ 同じ形 } で
\textbf{ 対応する頂点 } で抜き出すこと。

$^{**}$ 相似条件の確定には \textbf{ 結論 } を使ってはならない。
\newpage
