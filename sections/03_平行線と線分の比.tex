% 第3節 平行線と線分の比
\section{平行線と線分の比}

一般に、三角形と平行線については、次のことが成り立つ。

%平行線と線分の比
    \begin{theorembox}[thm:三角形と平行線の比①]{\textbf{三角形と平行線の比①}}
        \label{thm:三角形と平行線の比①}
        右の図の $\triangle ABC$ において

        \begin{enumerate}
            \item $DE /\!/ BC$ ならば \\
                $\fitblankbf[AD : AB = AE : AC = DE : BC]{40}$
            \item $DE /\!/ BC$ ならば \\
                $\fitblankbf[AD : DB = AE : EC]{40}$
        \end{enumerate}

        が成り立つ。
    \end{theorembox}

    Theorem\ref{thm:三角形と平行線の比①}は相似な図形を考えることによって簡単に証明できる。

    \begin{theorembox}[thm:平行線と線分の比]{\textbf{平行線と線分の比}}
        右の図において、3直線 $l, m, n$ が平行であるとき、
        \vspace{0.5em}

        $\fitblankbf[AB : BC = A'B' : B'C']{40}$

        が成り立つ。
        \vspace{7em}
    \end{theorembox}
    \vspace{1em}

    また、Theorem\ref{thm:三角形と平行線の比①}はその逆も成り立つ。

    \begin{theorembox}[thm:三角形と平行線の比②]{\textbf{三角形と平行線の比②}}
        \label{thm:三角形と平行線の比②}
        右の$\triangle ABC$ において

        \begin{enumerate}
            \item $AD : AB = AE : AC$ ならば \\
                $\fitblankbf[DB /\!/ EC]{40}$
            \item $AD : DB = AE : EC$ ならば \\
                $\fitblankbf[DE /\!/ BC]{40}$
        \end{enumerate}

        が成り立つ。
    \end{theorembox}
    \vspace{1em}
\newpage

    \begin{theorembox}[thm:三角形の内角の二等分線と比]{\textbf{三角形の内角の二等分線と比}}
        \label{thm:三角形の内角の二等分線と比}
        $\triangle ABC$ の $\angle BAC$ の二等分線と辺 $BC$ との交点を $D$ とすると
        \vspace{0.5em}

        $\fitblankbf[AB : AC = BD : CD]{40}$

        が成り立つ。
        \vspace{7em}
    \end{theorembox}
    \vspace{1em}

\begin{theorembox}[thm:三角形の外角の二等分線と比]{\textbf{三角形の外角の二等分線と比}}
    \label{thm:三角形の外角の二等分線と比}
    $AB \neq AC$ である $\triangle ABC$ の $\angle BAC$ の外角の二等分線と辺 $BC$ の延長との交点を $D$ とすると
    \vspace{0.5em}

    $\fitblankbf[AB : AC = BD : DC]{40}$

    が成り立つ。
    \vspace{7em}
\end{theorembox}


\newpage

