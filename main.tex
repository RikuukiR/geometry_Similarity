\documentclass[12pt,a4paper]{jsarticle}
\usepackage[margin=20truemm]{geometry}

% 節番号を「第1節」形式に
\renewcommand{\thesection}{第\arabic{section}節}

\input{preamble.tex}

% subsection番号を「1.1」形式に(thesectionの影響を受けないように)
\renewcommand{\thesubsection}{\arabic{section}.\arabic{subsection}}

% カウンターの初期値を設定
\setcounter{section}{0}
\setcounter{subsection}{0}
\setcounter{definition}{0}
\setcounter{theorem}{0}

\title{【中学数学】 図形と相似}                % タイトル
\author{}                     % 著者(空白でOK)
\date{}                       % 日付(空白)

\begin{document}
\begin{center}
{\LARGE 【中学数学】 第1章 図形と相似}

体系数学2/数研出版/幾何編
\end{center}

% --- 著者と日付を右寄せで表示 ---
\vspace{5mm}
\hfill Riku Sugawara \\
\hfill 11.2025

% ========================================
% 各セクションの読み込み
% ========================================

% 第1節 相似な図形
\section{相似な図形}

%相似
    \begin{definitionbox}[def:相似]{\textbf{相似}}
        2つの図形の一方の図形を \fitblankbf[拡大]{5} または\fitblankbf[縮小]{5} した図形が他方と合同になるとき、
        この2つの図形は \fitblankbf[相似]{5} であるという。
    \end{definitionbox}
    \vspace{1em}

    一言で言うと、大きさは \fitblankbf[異なる]{5} が、\fitblankbf[形は同じ]{8} の図形のことである。
    したがって、

    \begin{theorembox}[thm:相似な図形の性質]{\textbf{相似な図形の性質}}
        相似な図形の性質は以下の2つである。

    \begin{enumerate}
            \item 相似な図形では 対応する \fitblankbf[線分の長さの比]{10} はそれぞれ等しい。
            \item 相似な図形では 対応する \fitblankbf[角の大きさ]{8} はすべて等しい。
        \end{enumerate}
    \end{theorembox}
    \vspace{1em}

    \begin{definitionbox}[def:相似な図形の表し方]{\textbf{相似な図形の表し方}}
        2つの図形が相似であることを記号 $\fitblankbf[\sim]{3}$ を用いて表す。

        例えば、$\triangle ABC$ と $\triangle DEF$ が相似であることを $\fitblankbf[\triangle ABC \sim \triangle DEF]{18}$ と表し、
        「三角形 ABC 相似 三角形 DEF」と読む。
        また、記号 $\sim$ を用いるときは、頂点の順番を \fitblankbf[対応する順]{5} で書く。
    \end{definitionbox}
    \vspace{1em}

%相似比
    \begin{definitionbox}[def:相似比]{\textbf{相似比}}
        相似な図形で、対応する線分の長さの比を \fitblankbf[相似比]{5} という。
    \end{definitionbox}
    \vspace{1em}

\newpage


% 第2節 三角形の相似条件
\section{三角形の相似条件}

特に三角形においては、相似になるための条件が知られている。これを \fitblankbf[相似条件]{8} といい、次の3つがある。
相似条件は証明中に使い、\uwave{一言一句違わずに} 使わなければならない。

%相似条件
    \begin{theorembox}[thm:相似条件]{\textbf{相似条件}}
        2つの三角形が相似であるための条件は以下の3つである。

        \begin{enumerate}[itemsep=1em]
            \item \fitblankfixed[3組の辺の比がすべて等しい]{30}
            \item \fitblankfixed[2組の辺の比とその間の角がそれぞれ等しい]{30}
            \item \fitblankfixed[2組の角がそれぞれ等しい]{30}
        \end{enumerate}
    \end{theorembox}
    \vspace{1em}

\textbf{注意:} 合同条件と相似条件の違いに注意すること。以下、合同条件を書きなさい。

\begin{enumerate}[itemsep=1em]
    \item \fitblankfixed[3組の辺がそれぞれ等しい]{30}
    \item \fitblankfixed[2組の辺の比とその間の角がそれぞれ等しい]{30}
    \item \fitblankfixed[2組の角がそれぞれ等しい]{30}
\end{enumerate}
\vspace{1em}
\newpage

証明の流れと型を身につけてもらいたい。学校の型と異なるかとは思うが、ここでは私がオススメする型を紹介する。
特にこだわりがない場合は、この型を是非使ってもらいたい。

ここで使用する流れや型は2年次で学習した「合同な図形」の証明においても使用した汎用的なものである。

%証明の流れ
    \begin{theorembox}[thm:証明の流れ]{\textbf{証明の流れ}}
        \label{thm:証明の流れ}
        証明の流れは以下のようになる。

        \begin{enumerate}
            \item \fitblankbf[証明したい図形]{10} を確認し、一方を\fitblankbf[赤い三角形]{8}で、もう一方を\fitblankbf[青い三角形]{8}で囲む。
            \item 赤い三角形と青い三角形を \fitblankbf[抜き出し]{8} $^{*}$、\fitblankbf[仮定]{8} や \fitblankbf[条件]{8} を描き込む。
            \item(2)で抜き出した図形をもとに \fitblankbf[相似条件]{8} を確定させる。$^{**}$
        \end{enumerate}
    \end{theorembox}
    \vspace{0.5em}

    Theorem \ref{thm:証明の流れ} で証明の下準備は完了!証明は書き始める前に終えてなければならない。
    したがって、Theorem \ref{thm:証明の流れ} を使って証明の型にはめていくことになる。
    \vspace{0.5em}

    \begin{theorembox}[thm:証明の型]{\textbf{証明の型}}
        \label{thm:証明の型}
        証明の型は以下のようになる。ここでは \textcolor{red}{\underline{$\triangle ABC$}} と
        \textcolor{blue}{\underline{$\triangle DEF$}} が相似であることを証明する。

        \begin{proof}
            \setlength{\baselineskip}{1.5\baselineskip}
            \mbox{}\\
            \textcolor{red}{\fitblankbf[\triangle ABC]{8}} と \textcolor{blue}{\fitblankbf[\triangle DEF]{8}} において

            \vspace{0.3em}
            \textcolor{black}{\uwave{\hspace{2em}根拠①\hspace{3em}}} より、

            \vspace{0.3em}
            \makebox[\linewidth][s]{\textcolor{red}{\fitblankbf[AB]{8}} = \textcolor{blue}{\fitblankbf[DE]{8}} \hfill ・・・①}

            \vspace{0.3em}
            \textcolor{black}{\uwave{\hspace{2em}根拠②\hspace{3em}}} ので、

            \vspace{0.3em}
            \makebox[\linewidth][s]{\textcolor{red}{\fitblankbf[\angle BAC]{8}} = \textcolor{blue}{\fitblankbf[\angle EDF]{8}} \hfill ・・・②}

            \vspace{0.3em}
            \textcolor{black}{\uwave{\hspace{2em}根拠③\hspace{3em}}} ので、

            \vspace{0.3em}
            \makebox[\linewidth][s]{\textcolor{red}{\fitblankbf[CA]{8}} = \textcolor{blue}{\fitblankbf[FD]{8}} \hfill ・・・③}

            \vspace{0.3em}
            ①、②、③より、\textcolor{black}{\fitblankbf[相似条件]{15}} ので、

            \vspace{0.3em}
            \textcolor{red}{\fitblankbf[\triangle ABC]{8}} $\sim$ \textcolor{blue}{\fitblankbf[\triangle DEF]{8}}
        \end{proof}
    \end{theorembox}
\vfill

\noindent
\rule{\textwidth}{0.4pt}

\noindent
$^{*}$ 図形を抜き出す際には \textbf{ 形は気にせず } 抜き出してもらって構わない。しかし、2つの三角形とも \textbf{ 同じ形 } で
\textbf{ 対応する頂点 } で抜き出すこと。

$^{**}$ 相似条件の確定には \textbf{ 結論 } を使ってはならない。
\newpage

% 第3節 平行線と線分の比
\section{平行線と線分の比}

一般に、三角形と平行線については、次のことが成り立つ。

%平行線と線分の比
    \begin{theorembox}[thm:三角形と平行線の比①]{\textbf{三角形と平行線の比①}}
        \label{thm:三角形と平行線の比①}
        右の図の $\triangle ABC$ において

        \begin{enumerate}
            \item $DE /\!/ BC$ ならば \\
                $\fitblankbf[AD : AB = AE : AC = DE : BC]{40}$
            \item $DE /\!/ BC$ ならば \\
                $\fitblankbf[AD : DB = AE : EC]{40}$
        \end{enumerate}

        が成り立つ。
    \end{theorembox}

    Theorem\ref{thm:三角形と平行線の比①}は相似な図形を考えることによって簡単に証明できる。

    \begin{theorembox}[thm:平行線と線分の比]{\textbf{平行線と線分の比}}
        右の図において、3直線 $l, m, n$ が平行であるとき、
        \vspace{0.5em}

        $\fitblankbf[AB : BC = A'B' : B'C']{40}$

        が成り立つ。
        \vspace{7em}
    \end{theorembox}
    \vspace{1em}

    また、Theorem\ref{thm:三角形と平行線の比①}はその逆も成り立つ。

    \begin{theorembox}[thm:三角形と平行線の比②]{\textbf{三角形と平行線の比②}}
        \label{thm:三角形と平行線の比②}
        右の$\triangle ABC$ において

        \begin{enumerate}
            \item $AD : AB = AE : AC$ ならば \\
                $\fitblankbf[DB /\!/ EC]{40}$
            \item $AD : DB = AE : EC$ ならば \\
                $\fitblankbf[DE /\!/ BC]{40}$
        \end{enumerate}

        が成り立つ。
    \end{theorembox}
    \vspace{1em}
\newpage

    \begin{theorembox}[thm:三角形の内角の二等分線と比]{\textbf{三角形の内角の二等分線と比}}
        \label{thm:三角形の内角の二等分線と比}
        $\triangle ABC$ の $\angle BAC$ の二等分線と辺 $BC$ との交点を $D$ とすると
        \vspace{0.5em}

        $\fitblankbf[AB : AC = BD : CD]{40}$

        が成り立つ。
        \vspace{7em}
    \end{theorembox}
    \vspace{1em}

\begin{theorembox}[thm:三角形の外角の二等分線と比]{\textbf{三角形の外角の二等分線と比}}
    \label{thm:三角形の外角の二等分線と比}
    $AB \neq AC$ である $\triangle ABC$ の $\angle BAC$ の外角の二等分線と辺 $BC$ の延長との交点を $D$ とすると
    \vspace{0.5em}

    $\fitblankbf[AB : AC = BD : DC]{40}$

    が成り立つ。
    \vspace{7em}
\end{theorembox}


\newpage


% % 第4節 中点連結定理
\section{中点連結定理}

よく用いられる三角形の2辺の中点を結んだ線分の性質について、
ここまで学習したことを使うと次の定理が成り立つことがわかる。

%中点連結定理
    \begin{theorembox}[thm:中点連結定理]{\textbf{中点連結定理}}
        $\triangle ABC$ の辺 $AB$ と $AC$ の中点をそれぞれ $M$ , $N$ とすると

        \begin{enumerate}
            \item $\fitblankfixed[MN /\!/ BC]{15}$
            \item $\fitblankfixed[MN = \frac{1}{2} BC]{15}$
        \end{enumerate}

        が成り立つ。
        \vspace{5em}
    \end{theorembox}
    \vspace{1em}


\newpage

% % 第5節 相似な図形の面積比、体積比
\section{相似な図形の面積比、体積比}

相似な図形の面積比と体積比について学習する。

%面積比
    \begin{theorembox}[thm:相似な図形の面積比]{\textbf{相似な図形の面積比}}
        相似比が $\fitblankbf[m:n]{5}$ である2つの図形の面積比は $\fitblankbf[m^2:n^2]{8}$ である。
    \end{theorembox}
    \vspace{1em}

面積比を図で表すと以下のようになる。
\vspace{15em}

%体積比
    \begin{theorembox}[thm:相似な立体の体積比]{\textbf{相似な立体の体積比}}
        相似比が $\fitblankbf[m:n]{5}$ である2つの立体の体積比は $\fitblankbf[m^3:n^3]{8}$ である。
    \end{theorembox}
    \vspace{1em}

体積比を図で表すと以下のようになる。
\vspace{15em}

%例
    \begin{definitionbox}[def:面積比と体積比のまとめ]{\textbf{面積比と体積比のまとめ}}
        相似比が $m:n$ のとき、
        \begin{itemize}
            \item 面積比は $\fitblankbf[m^2:n^2]{8}$
            \item 体積比は $\fitblankbf[m^3:n^3]{8}$
        \end{itemize}
    \end{definitionbox}
    \vspace{1em}

\newpage



\end{document}

